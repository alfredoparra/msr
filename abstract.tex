Some decision theorists argue that when playing a prisoner's
dilemma-type game against a sufficiently similar opponent, we should
cooperate to make it more likely that our opponent also cooperates. This
idea, which Hofstadter calls superrationality, has strong implications
when combined with the insight from modern physics that we live in a
large universe or multiverse of some sort. If we care about what happens
in civilizations located elsewhere in the multiverse, we can
superrationally cooperate with some of the their inhabitants. That is,
if we take their values into account, this makes it more likely that
they do the same for us. In this paper, I attempt to assess the
practical implications of this idea. I argue that to reap the full gains
from trade, everyone should maximize the same impartially weighted sum
of the utility functions of all collaborators. I also argue that we can
obtain at least weak evidence about the content of these utility
functions. In practice, the application of superrationality implies that
we should promote causal cooperation, moral pluralism, moral reflection,
and ensure that our descendants, who will be smarter and thus better at
finding out how to benefit other superrationalists in the universe,
engage in superrational cooperation.

