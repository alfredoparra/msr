\documentclass[]{article}
\usepackage[margin=1.5in]{geometry}
\usepackage{lmodern}
\usepackage{amssymb,amsmath}
\usepackage[backend=biber,natbib=true,style=authoryear-comp,citestyle=authoryear]{biblatex}
\usepackage{ifxetex,ifluatex}
\usepackage{fixltx2e} % provides \textsubscript
\ifnum 0\ifxetex 1\fi\ifluatex 1\fi=0 % if pdftex
  \usepackage[T1]{fontenc}
  \usepackage[utf8]{inputenc}
\else % if luatex or xelatex
  \ifxetex
    \usepackage{mathspec}
  \else
    \usepackage{fontspec}
  \fi
  \defaultfontfeatures{Ligatures=TeX,Scale=MatchLowercase}
\fi
% use upquote if available, for straight quotes in verbatim environments
\IfFileExists{upquote.sty}{\usepackage{upquote}}{}
% use microtype if available
\IfFileExists{microtype.sty}{%
\usepackage[]{microtype}
\UseMicrotypeSet[protrusion]{basicmath} % disable protrusion for tt fonts
}{}
\PassOptionsToPackage{hyphens}{url} % url is loaded by hyperref
\usepackage[unicode=true]{hyperref}
\usepackage{xcolor}
\hypersetup{
            colorlinks,
            linkcolor={red!80!black},
		    citecolor={black},
		    urlcolor={blue!80!black},
            pdftitle={Values and acausal reasoning of whole brain emulations},
            pdfborder={0 0 0},
            breaklinks=true}
\urlstyle{same}  % don't use monospace font for urls
\IfFileExists{parskip.sty}{%
\usepackage{parskip}
}{% else
\setlength{\parindent}{0pt}
\setlength{\parskip}{6pt plus 2pt minus 1pt}
}
\setlength{\emergencystretch}{3em}  % prevent overfull lines
\providecommand{\tightlist}{%
  \setlength{\itemsep}{0pt}\setlength{\parskip}{0pt}}
\setcounter{secnumdepth}{0}
% Redefines (sub)paragraphs to behave more like sections
\ifx\paragraph\undefined\else
\let\oldparagraph\paragraph
\renewcommand{\paragraph}[1]{\oldparagraph{#1}\mbox{}}
\fi
\ifx\subparagraph\undefined\else
\let\oldsubparagraph\subparagraph
\renewcommand{\subparagraph}[1]{\oldsubparagraph{#1}\mbox{}}
\fi

% set default figure placement to htbp
\makeatletter
\def\fps@figure{htbp}
\makeatother

\bibliography{references_ems.bib}  % The name of your .bib file.

\title{Values and acausal reasoning of
whole brain emulations\\ \vspace{5mm} \small{Complementary notes on multiverse-wide superrationality}}
\author{Caspar Oesterheld}
\date{}

\begin{document}
\maketitle

An important step in the development of humanity and potentially other
civilizations in the multiverse with significant ramifications for
values could be
\href{https://en.wikipedia.org/wiki/Mind_uploading}{whole
brain emulation}, i.\,e. scanning a biological brain, uploading it into
a computer, and then simulating its behavior at a level of detail that
preserves the brain's functionality. To me, it seems likely that whole
brain emulation will be possible at some point, but it is plausible that
other transitions (like \emph{de novo} artificial intelligence, i.\,e.
artificial intelligent systems not modeled after humans or other
animals) will happen earlier\footnote{Discussion of whether \emph{de
  novo} artificial intelligence or whole brain emulation will arrive
  first is given in chapter 4, subsection ``Artificial Intelligence'' in
  Hanson's \emph{The Age of Em}, as well as section V in
  \href{http://slatestarcodex.com/2016/05/28/book-review-age-of-em/}{Scott
  Alexander's review of the book}. Of course, the two topics of AI and
  whole brain emulation timelines can also be discussed separately.
  Sandberg and Bostrom discuss whole brain emulation
  timelines~\citeyear{Sandberg2008-nd}; for overviews on AI timelines, see Luke Muehlhauser's
  \href{http://www.openphilanthropy.org/focus/global-catastrophic-risks/potential-risks-advanced-artificial-intelligence/ai-timelines}{What Do We Know about AI Timelines?}, and a
  \href{http://aiimpacts.org/ai-timeline-surveys/}{meta-survey at
  AI Impacts}.}, thereby precluding an era of whole brain
emulations. Following Robin Hanson, who has written extensively on the
topic, I will call the computer programs resulting from this process
\emph{ems}. Since (by assumption) they are functionally equivalent to
human brains, ems can do most of the things that humans can do. They can
learn, work, form relationships, invent things, walk around in robotic
bodies, etc. While the lack of a biological body does place some
limitations on what ems can do -- they cannot, for instance, have
regular children -- many of the sensations associated with physical
activities (the taste of food, sex, etc.) can presumably be simulated.
They can, however, also do many things beyond the limits of human
capabilities. Being software, ems can be copied an indefinite number of
times. By moving an em to a faster computer (or giving them more CPU
time on their current one), one can manipulate their thinking speed.
They can also travel at incredible speed and low cost via the Internet
and other digital communication networks.

The advantages that ems have over biological humans plausibly make it
impossible for humans to compete with them. Ems do not get sick, nor do
they require payment for the costs of food, health insurance, and so on.
Moreover, if one scans a few elite (i.\,e. smart, diligent, reliable,
educated, hard-working, etc.) humans and creates millions of copies of
the resulting ems, then everyone but the most competent humans will be
unable to compete with them. The em economy will also be able to grow
much faster than its human counterpart by virtue of not being bounded by
labor at all. Once you have one em and it is cheap to build its required
hardware, you can easily build more of them. Producing more human
workers, on the other hand, takes a lot of time and resources. The
consequences of an em-dominated society (as well as the arguments for
it) are detailed in Robin Hanson's book 
\href{http://ageofem.com/}{The Age of Em} \citeyear{Hanson2016-yn}. For shorter
introductions to the topic, see:

\begin{itemize}
\item
  Section III of Scott Alexander's
  \href{http://slatestarcodex.com/2016/05/28/book-review-age-of-em/}{review
  of Hanson's book};
\item
  \href{https://www.youtube.com/watch?v=9qcIsjrHENU}{Robin
  Hanson's TEDx talk};
\item
  the \href{http://ageofem.com/}{book's website}, which contains
  short summaries of all chapters.
\end{itemize}

We will only discuss the two issues which seem most relevant to
multiverse-wide superrationality: Will ems take superrationality
seriously and what will their values be?

\subparagraph{Ems and decision theory}\label{ems-and-decision-theory}

Starting with decision theory, I see good reasons to assume that ems
will take non-causal decision theory more seriously than humans do.
Whereas Newcomb's problem and prisoner's dilemmas with replicas are pure
thought experiments for humans -- some even argue that they cannot be
set up at all \parencite{Binmore2007-mk} -- they can actually be implemented with ems
\parencite{Yudkowsky2010-xo}. For example, one could take
one em, create 19 copies of her, and then have them play a donation
game. Although such thought experiments are unlikely to be part of the
day-to-day lives of ems, reasoning about correlations between copies may
become much more useful in practice, too. Whereas twins are rare and
often differ significantly in by their mid-twenties in terms of
experiences, most ems will have many copies (Hanson 2016, p. 155)
including several recent ones (ibid., pp. 169ff.). Correlations between
copies who until recently shared all experiences will be especially
strong. Because copies can trust each other more, it seems plausible
that they will interact with each other a lot, potentially trying to
coordinate as ``clans'' (ibid., pp. 227f.). Overall, this means that
near-copies will often interact with each other, and it thus seems
plausible that they would quickly learn to use non-causal decision
theories to their advantage.

While we should expect correlated decision-making to be more relevant
for ems than for humans, many of the arguments I outline in section
``Superrational cooperation on Earth'' of \emph{Multiverse-wide Cooperation via
Correlated Decision Making} 
%TODOLaTeX: link
nevertheless dampen the importance of em
superrationality, possibly to the extent that it may not be very
important in practice after all.

\subparagraph{Em era values}\label{em-era-values}

We now turn to the values of ems. The following list is expanded from
the
\href{https://casparoesterheld.com/2016/08/30/the-age-of-em-summary-of-policy-relevant-information/\#Values}{section
on values} from a summary of \emph{The Age of Em} I published in my blog.
I should note that I have some reservations about Hanson's predictions
in this area, because
\href{https://casparoesterheld.com/2016/08/30/the-age-of-em-summary-of-policy-relevant-information/\#regulation}{I
tentatively expect higher regulation} than Hanson does. Needless to say
(and as Hanson acknowledges), making predictions about em-era values on
Earth is already very difficult, and attempting to transfer them to
other civilizations makes this even harder. Hence, these arguments
should be interpreted as shifting probabilities by single percentage
points at most.

\begin{itemize}
\item
  Ems have no reasons to farm animals for food or use them for testing
  drugs.
  \href{https://en.wikipedia.org/wiki/Cognitive_dissonance}{Cognitive
  dissonance} theory suggests that this will make ems care about
  animals more than humans do.
\item
  As opposed to most humans, em copies will mostly be created on demand.
  That is: if you are an em, you apply for jobs (or employers offer them
  to you) and for every job that you get, you create a copy that fills
  that particular job. (In some unregulated dystopian scenarios it is
  also possible that ems cannot veto on whether they want to have a copy
  made of themselves.) This means that the question of ``will this
  specific life be worth living?'' will be more salient to ems than
  humans, who can rarely predict what their children's lives will be
  like. They will also feel more responsible for having made the
  decision to live their lives (given that the decision was made by a
  copy), so they are less likely to
  \href{https://en.wikipedia.org/wiki/Antinatalism}{resent their
  creation} (ibid, p. 120).\footnote{It is, however, still possible.
    Incidentally, the critically acclaimed science-fiction novel
    \href{https://en.wikipedia.org/wiki/Permutation_City}{Permutation
    City} by Greg Egan starts with a newly created copy resenting its
    existence.} Also, there is strong selection pressure favoring ems
  who consider, say, a life without much leisure to be positive (p.
  123). Overall, there are selection pressures towards ems wanting to
  make many copies of themselves.
\item
  There is a strong selection pressure against ems who are not willing
  to create \emph{short-lived} copies of themselves. If competition is
  strong enough (and human nature sufficiently flexible), ems will
  probably still value that at least one of their copies will survive,
  but they would probably not disvalue the death of individual copies
  that much. This could lead to a moral view wherein copy clans, rather
  than individuals, count as the morally relevant entities. This would
  be similar to how many people care about protecting species rather
  than (and often at the cost of) preserving individuals.
\item
  Hanson argues that ems will probably not suffer much (p. 153, 371),
  because their virtual reality (and even their own brain) can be so
  easily controlled. Given that experiencing suffering
  \href{http://reducing-suffering.org/how-important-is-experiencing-suffering-for-caring-about-suffering/}{\emph{probably
  correlates} \emph{with caring about suffering}}, this could mean that
  ems will care less about the suffering of others.
\item
  Assuming that individual aspects of ems can be tweaked, they could be
  made especially thoughtful, friendly, and so on (p. 150).
\item
  Because of higher competition, ems will work more (e.\,g. see pp.
  167ff., 207) and be paid less. Hence, they will not have the resources
  for altruistic activities that modern elites currently have.
\item
  People who are more productive tend to be married, intelligent,
  extroverted, conscientious and non-neurotic. Smarter people are more
  cooperative, patient, rational and law-abiding, and also tend to favor
  trading with foreigners. Because ems will be
  \href{https://casparoesterheld.com/2016/08/30/the-age-of-em-summary-of-policy-relevant-information/\#WhoBecomesEm}{selected
  for productivity}, they will tend to have these traits as well (p.
  163).

  \begin{itemize}
  \item
    It is somewhat unclear whether ems will be more or less religious.
    Apparently religious people are more productive, but they are also
    less innovative (p. 276, 311). Hanson expects that religions will be
    able to adapt to the em world (p. 312).
  \end{itemize}
\item
  Workaholics tend to be male and males are also more competitive, so
  the em society may consist mostly of males (p. 167).
\item
  Due to the possibility of creating a lot of copies when an em reaches
  a particular age, only to destroy most of the copies later, most ems
  will likely be at the peak productivity age of around 40 to 50 or
  older (p. 202ff.). 50-year-olds tend to be less supportive of war than
  younger people (p. 250). Also, ``older people tend to associate
  happiness more with peacefulness, as opposed to excitement.'' (p. 205)
\item
  Most ems will not have children (p. 211f.), which
  \href{http://www.theatlantic.com/health/archive/2015/11/having-kids-can-make-parents-less-empathetic/416592/}{could,
  among other things, make them more compassionate towards others}
  \parencite{Gilead2014-rv}.
\item
  At some point, it may become attractive to scan children in order to
  turn them into ems, since they can then adapt more easily to the em
  world (p. 212). This could give an advantage to ruthless countries and
  children of psychopathic parents,
  \href{http://aftermath-surviving-psychopathy.org/index.php/is-psychopathy-genetic/}{who
  are themselves more likely to be psychopathic}
  \parencite{Viding2010-my,Waldman2006-ka,Farrington2006-az}.
\item
  Space will lose some appeal, because it takes ages of subjective time
  to travel there (p. 225).
\item
  If male ems are ``castrated'' (however that would exactly work for
  ems) because of the gender imbalances and the obsoleteness of sexual
  reproduction, then they will tend to be more sympathetic (p. 285).
\item
  ``Ems can travel more cheaply to virtual nature parks, and need have
  little fear that killing nature will somehow kill them.'' (p. 303) If
  we assume the latter to be a main reason for humans' care for the
  environment
  \href{http://www.collectionscanada.gc.ca/obj/s4/f2/dsk3/ftp04/NQ59050.pdf\#page=165}{(Woodcock,
  2000, ch. 4.IV, section ``Ecosystems are Inherently Valuable?'')}, we
  should expect ems to care significantly less about preserving nature.
\item
  The classic targets of charity -- alms, schools and hospitals -- may
  all be less necessary in an em society (p. 302). This may lead ems to
  support other kinds of charity.
\item
  ``New em copies and their teams are typically created in response to
  new job opportunities. Such teams typically end or retire when these
  jobs are completed. Thus ems are likely to identify strongly with
  their particular jobs; their jobs are literally their reason for
  existing.'' (p. 306, also see p. 328) Maybe this implies that ems will
  be less involved in pursuing ethical causes to give their life a
  meaning.
\item
  Ems are far more likely to be anti-substratist than humans
  (obviously).
\item
  Ems may find it more natural to view consciousness as a matter of
  degrees rather than absolutes, seeing as em minds will differ in speed
  and some may exist only as partial minds (p. 341ff.).
\item
  ``{[}Because ems will be poorer than citizens of industrialized
  societies, they{]} seem likely to return to conservative (farmer)
  cultural values, relative to liberal (forager) cultural values.
  {[}...{]} Today, liberals tend to be more open-minded, creative,
  curious, and novelty seeking, while conservatives tend to be more
  orderly, conventional, and organized. If, relative to us, ems prefer
  farmer-like values to forager-like values, then ems more value things
  such as self-sacrifice, self-control, religion, patriotism, marriage,
  politeness, material possessions, and hard work, and less value
  self-expression, self-direction, tolerance, pleasure, nature, novelty,
  travel, art, music, stories, and political participation. {[}...{]} If
  ems are indeed more farmer-like, they tend to envy less, and to more
  accept authority and hierarchy, including hereditary elites and
  ranking by gender, age, and class. They are more comfortable with war,
  discipline, bragging, and material inequalities, and push less for
  sharing and redistribution. They are less bothered by violence and
  domination toward the historical targets of such conflicts, including
  foreigners, children, slaves, animals, and nature. {[}...{]}
  Farmer-like ems have a stronger sense of honor and shame, enforce more
  conformity and social rules, and care more for cleanliness and order
  \href{http://citeseerx.ist.psu.edu/viewdoc/download?doi=10.1.1.825.5270\&rep=rep1\&type=pdf}{(Stern
  et al. 2014)}.'' (pp. 326-328.)
\item
  ``As ems have near subsistence (although hardly miserable) income
  levels, and as wealth levels seem to cause cultural changes, we should
  expect em culture values to be more like those of poor nations today.
  As Eastern cultures grow faster today, and as they may be more common
  in denser areas, em values may be more likely to be like those of
  Eastern nations today.'' Citing
  \href{http://indicatorsinfo.pbworks.com/f/Inglehart+Mass+Priorities+and+Democracy.pdf}{Inglehart
  and Welzel (2010)} and
  \href{http://citeseerx.ist.psu.edu/viewdoc/download?doi=10.1.1.832.177\&rep=rep1\&type=pdf}{Schwartz
  et al. (2012)}, Hanson goes on: ``Together, these suggest that em
  cultures tend to value technology, money, hard work, and state
  intervention. They may also suggest that em culture values
  achievement, determination, thrift, authority, good and evil, and
  local job protection.'' (p. 322f.)
\item
  The em world will be dominated by a few (i.\,e. something like one
  thousand) copy clans, copied from humans who will tend to be selected
  for their eliteness. ``Today, Jews comprise a disproportionate
  fraction of extreme elites such as billionaires, and winners of prizes
  such as the Pulitzer, Oscar, and Nobel prizes
  \href{http://www.forbes.co.il/news/new.aspx?pn6Vq=J\&0r9VQ=IEII}{(Forbes
  2013)}. (I have sought but failed to find work identifying other
  elite ethnicities.) This weakly suggests that Jews are also
  disproportionately represented among ems.''
\end{itemize}

\begin{sloppypar} % Prevents URLs from exceeding page width
\printbibliography
\end{sloppypar}
\end{document}
