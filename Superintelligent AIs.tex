\documentclass[]{article}
\usepackage[margin=1.5in]{geometry}
\usepackage{lmodern}
\usepackage{amssymb,amsmath}
\usepackage[backend=biber,natbib=true,style=authoryear-comp,citestyle=authoryear]{biblatex}
\usepackage{ifxetex,ifluatex}
\usepackage{fixltx2e} % provides \textsubscript
\ifnum 0\ifxetex 1\fi\ifluatex 1\fi=0 % if pdftex
  \usepackage[T1]{fontenc}
  \usepackage[utf8]{inputenc}
\else % if luatex or xelatex
  \ifxetex
    \usepackage{mathspec}
  \else
    \usepackage{fontspec}
  \fi
  \defaultfontfeatures{Ligatures=TeX,Scale=MatchLowercase}
\fi
% use upquote if available, for straight quotes in verbatim environments
\IfFileExists{upquote.sty}{\usepackage{upquote}}{}
% use microtype if available
\IfFileExists{microtype.sty}{%
\usepackage[]{microtype}
\UseMicrotypeSet[protrusion]{basicmath} % disable protrusion for tt fonts
}{}
\PassOptionsToPackage{hyphens}{url} % url is loaded by hyperref
\usepackage[unicode=true]{hyperref}
\usepackage{xcolor}
\hypersetup{
            colorlinks,
            linkcolor={red!80!black},
		    citecolor={black},
		    urlcolor={blue!80!black},
            pdftitle={Values and non-causal reasoning of superintelligent AIs},
            pdfborder={0 0 0},
            breaklinks=true}
\urlstyle{same}  % don't use monospace font for urls
\IfFileExists{parskip.sty}{%
\usepackage{parskip}
}{% else
\setlength{\parindent}{0pt}
\setlength{\parskip}{6pt plus 2pt minus 1pt}
}
\setlength{\emergencystretch}{3em}  % prevent overfull lines
\providecommand{\tightlist}{%
  \setlength{\itemsep}{0pt}\setlength{\parskip}{0pt}}
\setcounter{secnumdepth}{0}
% Redefines (sub)paragraphs to behave more like sections
\ifx\paragraph\undefined\else
\let\oldparagraph\paragraph
\renewcommand{\paragraph}[1]{\oldparagraph{#1}\mbox{}}
\fi
\ifx\subparagraph\undefined\else
\let\oldsubparagraph\subparagraph
\renewcommand{\subparagraph}[1]{\oldsubparagraph{#1}\mbox{}}
\fi

% set default figure placement to htbp
\makeatletter
\def\fps@figure{htbp}
\makeatother

\bibliography{references_ai.bib}  % The name of your .bib file.

\title{Values and non-causal reasoning of superintelligent AIs \\ \vspace{5mm} \small{Complementary notes on multiverse-wide superrationality}}
\author{Caspar Oesterheld}
\date{}

\begin{document}
\maketitle

Because evolved minds (whether
\href{https://en.wikipedia.org/wiki/Mind_uploading}{uploaded or
not}) are not well-suited to frequent
\href{https://en.wikipedia.org/wiki/Software_maintenance}{maintenance}
or improvement, I expect many long-lasting civilizations to eventually
create artificial intelligence (AI) of a higher level of intelligence.
Readers who have yet to read about the topic are recommended to do so.
Of the many introductory texts on superintelligent AI, I can personally
recommend the following three:

\begin{itemize}
\item
  Tim Urban's highly accessible two-part introduction
  (\href{http://waitbutwhy.com/2015/01/artificial-intelligence-revolution-1.html}{part
  1},
  \href{http://waitbutwhy.com/2015/01/artificial-intelligence-revolution-2.html}{part
  2}) on \href{http://waitbutwhy.com/}{Wait But Why},
\item
  Nick Bostrom's more elaborate and academic book
  \href{https://en.wikipedia.org/wiki/Superintelligence:_Paths,_Dangers,_Strategies}{Superintelligence},
  and
\item
  Lukas Gloor's essay entitled
  \href{https://foundational-research.org/altruists-should-prioritize-artificial-intelligence/}{Altruists
  Should Prioritize Artificial Intelligence}.
\end{itemize}

In this piece, we assume some basic knowledge and only discuss the two
questions most relevant for multiverse-wide superrationality: do AIs
behave superrationally, i.\,e. do their decisions correlate with ours? And
what values would superrational AIs have?

\section{AIs and superrationality}\label{ais-and-superrationality}

Acausal decision theorists have not yet settled on a formal decision
theory that could be implemented in an AI (see section ``A short survey
of decision theories and their relation to superrationality'' in
\emph{Multiverse-wide Cooperation via Correlated Decision-Making}).
%TODOLaTeX: link
Nevertheless, it seems plausible that AIs will use a decision
theory that allows for acausal reasoning in a proper way. Indeed,
because AIs have a readable and copyable source code, acausal reasoning
may well be more relevant for them than for other agents.

However, some additional considerations should be taken into account
when thinking about behaving beneficially toward AIs. For one, the
correlation between human minds and AIs may be significantly weaker than
that with the other evolved minds, and this would imply that behaving in
a friendly way toward AIs is less important than doing so towards
evolved minds (see section ``Only helping superrational cooperators
helps you superrationally'' in \emph{Multiverse-wide Cooperation via
Correlated Decision-Making}).
%TODOLaTeX: link
Besides AI minds being designed rather than evolved, one could argue that superintelligent AIs will
simply know a lot more than we ever will. Perhaps they will even know,
in great detail, how we behave and hence not need to cooperate in order
to make our cooperation more likely. But as explained in section
``Lack
of knowledge is evidential power, part II: taking a step back'' of
\emph{Multiverse-wide Cooperation via Correlated Decision-Making}, 
%TODOLaTeX: link
any rational
agent should precommit before learning about the other agent's level of
cooperation (cf. section
``No reciprocity needed -- Whom to treat
beneficially'' in \emph{Multiverse-wide Cooperation
via Superrationality}).
%TODOLaTeX: link 
AIs will probably learn and think about superrationality before they can use their superior knowledge of
evolutionary psychology to assess the fraction of humans in the
multiverse who cooperate. At that point, it should be optimal for the AI
to precommit to legitimately cooperative behavior so that the other
agents in the multiverse make similar precommitments.

Overall, I think we should attempt to cooperate superrationally with
AIs.

\section{The values of superintelligent AIs}
\label{the-values-of-superintelligent-ais}

Two competing factors appear to be driving AI values: the values of
their creators and technical feasibility. Since the first
superintelligent AIs in any evolutionary lineage are created by evolved
minds, we should expect their values to reflect those of their evolved
creators. However, there are various reasons to assume that it is
difficult to make an AI do what one really wants. This suggests that at
least some civilizations will either fail to value-align their
superintelligent AI or resort to suboptimal solutions.

Any particular problem in value alignment has specific, yet
hard-to-predict implications for how value loading might fail or how
civilizations will attempt to solve value loading. For example,
\href{https://wiki.lesswrong.com/wiki/Complexity_of_value}{complexity
of value} suggests that most AIs will not actually hold their evolved
creators' values, but rather some (simple) preprogrammed approximation
or some indirect specification based on value learning (see
\cite{Bostrom2014-ay}). As another example, the problem of
\href{https://arbital.com/p/probable_environment_hacking/}{anthropic capture or probable
environment hacking} suggests that indirect specifications of values are less common and that
``mean'' value systems (i.\,e. ones that imply a willingness to hack other AI's probable environment)
will have more resources than one would otherwise expect.  Some failure modes (like
\href{https://casparoesterheld.com/2016/07/08/wireheading/}{wireheading}) would also make an
AI care little about what happens elsewhere in the multiverse.

The field of AI safety is still in its infancy. While I hope to have
illustrated how we can make principled guesses about the values of
superintelligent AIs in the multiverse in principle, this infancy makes
it hard to know how AI values will differ from the values of evolved
beings.

\begin{sloppypar} % Prevents URLs from exceeding page width
\printbibliography
\end{sloppypar}
\end{document}
